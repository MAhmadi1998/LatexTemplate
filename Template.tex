\documentclass[12pt,a4paper]{report}
\usepackage{fancyhdr}
\usepackage{amsmath, amssymb}
\usepackage{graphicx}
\usepackage{indentfirst}
\usepackage{xepersian}

\settextfont[Scale = 1.16667]{B Yas}
\setlatintextfont{Times New Roman}
\setcounter{secnumdepth}{0}

\pagestyle{fancy}
\renewcommand{\sectionmark}[1]{\markboth{#1}{}} % MAKE IT 'subsectionmark' INSTEAD OF 'sectionmark'

\fancyhf{}

\fancyhead[R]{\leftmark} %%%%%%%%%CHANGE 'L' TO 'R' HERE %%%%%%
\fancyfoot[C]{\thepage}
\fancypagestyle{plain}{%
    \fancyhf{}%
    \renewcommand{\headrulewidth}{0pt}%
}


\begin{document}
\title{بسم الله الرحمن الرحیم \\عنوان گزارش را در اینجا وارد کنید}
\author{ نام و نام خانوادگی \\ شماره دانشجویی} 
\maketitle

\tableofcontents

\newpage

\section{چکیده}
چکیده گزارش خود را در اینجا وارد کنید (اگر وجود دارد. )

\newpage

\section{مقدمه}

مقدمه را در اینجا وارد کنید

\newpage

\section{•}
سایر سکشن‌های لازم را به راحتی با دستور فوق می‌توانید اجرا کنید

\section{قرار دادن عکس}
قطعه کد قرار دادن عکس در گزارش
\begin{center}
	\begin{figure}[h]
	\centering
	\includegraphics[scale = .1]{1.png}
    \caption{عکس قرار داده شده}
    \label{a}
\end{figure}
\end{center}
تصویر \ref{a} تصویری از آرم دانشگاه تهران است. 
\newpage 

\section{وارد کردن فرمول‌های ریاضی }
برای وارد کردن فرمول‌های ریاضی به طور کلی دو روش وجود دارد که عبارتند از: \lr{inline} و \lr{display} .
\\
رابطه \lr{$E=mc^2$} معروف ترین رابطه انیشتین است.
\\
برای محاسبه نیروی وارد بر یک جسم شتابدار از رابطه زیر استفاده می‌کنیم:
\begin{equation}
	\label{Fma}
	F=ma.
\end{equation}
ما رابطه \eqref{Fma} را دوست داریم.
همچنین برای مجاسبه انرژی پتانسیل داریم:
\begin{equation}
	U=mgh.
\end{equation}

\begin{equation}\label{MyEq}
  \frac{a}{b} \times \frac{b}{c} = \frac{a \times b}{b \times c} = \frac{a}{c}
\end{equation}


\newpage
\section{مراجع}
در این قسمت مراجع خود را وارد کنید.




\end{document} 